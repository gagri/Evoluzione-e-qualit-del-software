\section{Code Smell}
Il termine Code Smell è stato coniato per la prima volta da Kent Beck e nell’Ingegneria del Software indica codice sorgente che, nonostante sia funzionante, presenta delle imperfezioni logiche che ne riducono la qualità. Non si tratta di veri e propri errori ma la loro presenza può rallentare la manutenzione del software e rendere difficile la  scoperta di bug. \\
Quando ci si accinge a progettare un nuovo software è conveniente definirne una struttura facilmente comprensibile, in maniera tale da garantire flessibilità in caso di aggiunte al codice da effettuare in un secondo momento. I Code Smells incrementano il Software decay, cioè il decadimento del Software a causa della poca manutenzione effettuabile su di esso. La metodologia che permette la correzione di questi prende il nome di Refactoring. 
%I code smell sono parti di codice sorgente caratterizzate da difetti di programmazione; non sono definibili come veri e propri errori. La loro presenza non causa problemi direttamente al funzionamento del software, però possono rallentare la manutenzione del software e rendere difficile la  scoperta di bug. 
SonarCloud consente, in base alle regole settate nel quality profile, di individuare diverse tipologie di smells come ad esempio:
\begin{itemize}
\item "==" and "!=" should not be used when "equals" is overridden
\item "@CheckForNull" or "@Nullable" should not be used on primitive types
\item "@Deprecated" code should not be used
\item "@EnableAutoConfiguration" should be fine-tuned
\item "@Import"s should be preferred to "@ComponentScan"s
\item "@Override" should be used on overriding and implementing methods
\end{itemize}
Ad ognuno viene associato un diverso tipo di severità, in particolare si ha:
\begin{enumerate}
\item blocker;
\item major;
\item info;
\item critical;
\item minor.
\end{enumerate}
In base alla severità viene associato un certo Technical Debt.