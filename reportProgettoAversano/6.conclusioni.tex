\chapter{Conclusioni}\label{cap5}
In base all'analisi effettuata è possibile evidenziare i seguenti risultati: \\ \\
\textbf{Come varia il numero di cloni nelle versioni successive di uno stesso progetto?}\\ 
I risultati hanno evidenziato che il numero di cloni tende ad aumentare nelle versioni successive di uno stesso progetto. Si è notato che l'unico dato anomalo si ha sulla versione di JabRef in cui il numero di cloni diminuisce rispetto alla versione precedente. Questo risultato non va però ad inficiare il trend verificato nelle analisi degli altri progetti in quanto si nota che nella versione $4.3$ vi è non solo un aumento rispetto alla versione $4.2$, ma addirittura un incremento rispetto a tutte le precedenti versioni. \\ \\
\textbf{Quale tipo di clone prevale?}\\
I risultati hanno evidenziato che in tutti i progetti analizzati il numero di cloni di tipo 3 è predominante. Inoltre, i cloni ti tipo 2 risultano maggiori rispetto a quelli di tipo 1.\\ \\
\textbf{Come sono distribuiti i cloni nel progetto?}\\ 
Analizzando le percentuali di file con e senza cloni si evidenzia una maggiore percentuale di file senza cloni. \\ \\
\textbf{Come variano i code smells nelle classi con e senza codice clonato?}\\
Si è visto che nei file contenenti cloni, il numero medio di code smell è in generale maggiore rispetto al caso in cui il file non contiene i cloni stessi. La differenza risulta maggiormente evidente in \textbf{FastJson}. Questo porta a concludere che la presenza di cloni ha un significativo impatto negativo sulla qualità del codice. \\ \\
\textbf{Come varia il technical debt nelle classi con e senza codice clonato?}\\
I risultati hanno mostrato che nei file contenenti cloni, il technical debt risulta mediamente maggiore. Quindi, il tempo necessario per gestire e manutenere una classe in cui sono presenti dei cloni, è più alto rispetto alle classi in cui non sono presenti. \newpage
\textbf{Qual è la relazione tra technical debt e code smells in classi con e senza cloni?}\\
I risultati hanno mostrato che nelle classi in cui sono presenti cloni, il rapporto tra tra code smell e techical debt è maggiore. Questo implica che è necessario più tempo per manutenere classi in cui sono presenti cloni oltre ad altri code smells.\\ \\
\textbf{Il tipo del clone impatta sulla dimensione del clone stesso?}\\
I risultati dell'analisi statistica hanno mostrato che mediamente i cloni di tipo 3 hanno numero di LOC maggiore, fatta eccezione per DNSJava. \\ \\
In conclusione è possibile affermare che, in generale, la presenza di cloni impatta in maniera negativa sulla qualità del codice rendendo quindi più complessa la manutenzione e l'evoluzione del software.