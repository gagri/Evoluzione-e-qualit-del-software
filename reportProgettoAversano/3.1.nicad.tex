\section{Nicad}
NiCad Clone Detector è uno strumento scalabile e flessibile per l'individuzione dei cloni basato su Txl e progettato per implementare il NiCad hybrid clone detection method. Quest'ultimo è un metodo ibrido che combina language-sensitive parsing con language-independent similarity analysis per identificare cloni significativi.
Il metodo comprende tre fasi principali: il parsing, la normalizzazione e il confronto.
Nella prima fase le sorgenti di input vengono parsate per poter estrarre tutti i frammenti, come ad esempio funzioni o blocchi, di una data granularità. Ogni frammento estratto è un potenziale clone e su di esso vengono effettuate alcune operazioni: spaziature e interruzioni di linee vengono normalizzati e i commenti vengono rimossi. 
Nella seconda fase, i frammenti estratti vengono normalizzati, filtrati o astratti prima del confronto. Ad esempio, possono essere trasformati rinominandoli o rimuomendo le dichiarazioni.
Nell'ultima fase, i frammenti normalizzati ed estratti sono confrontati a livello lineare usando un algoritmo LCS (Longest Common Subsequence) ottimizzato per rilevare i frammenti simili, ovvero i cloni. Il confronto è parametrizzato utilizzando una soglia di differenza che consente un rilevamento più o meno accurato. Ad esempio, una soglia di differenza 0.0 rileva solo i cloni esatti, 0.1 rileva quelli che possono differire del 10\%, 0.2 fino al 20\% e così via.

Nicad Clone detector è un'implementazione open source del metodo NiCad progettato per essere flessibile, estensibile e incorporabile negli IDE e nelle altre applicazioni. E'\ progettato per essere utilizzato via linea di comando Linux, Solaris, Cygwin o MAc OSs definendo la granularità desiderata dell'elaborazione e la directory del sorgente da analizzare, per esempio: 

\begin{center}
\verb|nicad functions java examples/dnsjava-2.1.8|
\end{center}

Nicad supporta due granularità, ossia funzioni (functions) e blocchi (blocks) e cinque linguaggi: C, C\#, Java, Phyton e WSDL.
Di default effettua il confronto senza normalizzare, filtrare o ridenominare trovando semplicemente cloni esatti o near-miss a quattro soglie di differenza: 0.0, 0.1, 0.2, e 0.3 che corrisponde allo 0\%, 10\%, 20\% e 30\% di linee diverse nei frammenti estratti.
L'output è costituito da un file XML ed un file HTML, memorizzati nella stessa cartella in cui è presente il sorgente. 
Per specificare la normalizzazione, il filtraggio e la ridenominazione si aggiunge il nome del file di configurazione a linea di comando:

\begin{center}
\verb|nicad functions java examples/dnsjava-2.1.8 type2|
\end{center}

In questo caso, si utilizza ad utilizzare il file di configurazione "type2.cfg".
I file di configurazione consentono all'utente di specificare una gamma di opzioni, come ad esempio la dimensione massima o minima dei cloni, la ridenominazione o le normalizzazioni da eseguire. Ogni opzione specificata in questi file invoca un plugin NiCad che viene eseguito prima del confronto tra i potenziali cloni. I plugin sono implementati in TXL e l'utente può anche aggiungere nuove normalizzazioni programmandole con questo linguaggio.






