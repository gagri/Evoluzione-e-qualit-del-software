\section{Elaborazione Dati SonarCloud}
La quantita' di informazioni e metriche fornite da SonarCloud risulta enorme rispetto a quelle che possono essere sfruttate al fine dell'analisi d'interesse. Per questo motivo a partire dalle analisi effettuate da SonarCloud si sono prelevate solo le due metriche d'interesse:
\begin{itemize}
	\item \textbf{Il Technical Debt} 
	\item \textbf{I Code Smell}
\end{itemize}
Tali informazioni sono legate al singolo file del sistema, e per questo e' possibile associarle all'analisi fornita da Nicad, nelle quali ad ogni clone e' associato il file su cui si trova. Per ogni sistema d'interesse, si e' proceduto ad analizzare le versioni selezionate in maniera indipendente. Il prelievo effettivo delle informazioni e' avventuo attraverso un semplice copia e incolla su un file di testo. Si e' preferito utilizzare questa procedura per via della semplicita' di informazioni da recuperare, sebbene la WebAPI di SonarCloud consenta il prelievo di informazioni aggregate.
In particolare sono stati creati due file testuali, uno contente la lista del path dei file del sistema  e il rispettivo Technical Debt, mentre l'altro associa i Code Smell e i path dei file.