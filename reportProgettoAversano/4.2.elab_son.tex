\section{Elaborazione Dati SonarCloud}
SonarCloud consente di estrarre una grande varietà di metriche ma, ai fini della nostra analisi, verranno prese in considerazione solo due cioè:
\begin{itemize}
	\item \textbf{Il Technical Debt} 
	\item \textbf{I Code Smell}
\end{itemize}
Tali informazioni sono legate alla singola classe del sistema e per questo è possibile associarle all'analisi fornita da Nicad. Per ogni sistema d'interesse si è proceduto ad analizzare le versioni selezionate in maniera indipendente. Il prelievo effettivo delle informazioni è avvenuto copiando e incollando i dati su un file di testo. Si è preferito utilizzare questa procedura per via della semplicità di informazioni da recuperare, sebbene la WebAPI di SonarCloud consenta il prelievo di informazioni aggregate.
Sono stati creati due file di testo: uno contente la lista dei path dei file del sistema e il rispettivo Technical Debt, mentre l'altro associa i Code Smell ai path dei file. Attraverso l'uso di due parser\footnote{ParserDebt.py e ParserSmell.py} è stato realizzato un file CSV che associa ad ogni file di ogni versione dei sistemi analizzati i corrispettivi Technical Debt e Code Smell.