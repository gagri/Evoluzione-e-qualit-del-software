\chapter{Introduzione}
I ricercatori hanno studiato a lungo gli effetti che hanno i cloni sulla gestione, manutenzione ed evoluzione dei sistemi software concludendo che i cloni hanno sia effetti positivi che negativi. Sulla scia di questi studi, in questo elaborato, si andrà ad analizzare l'effetto che hanno i cloni sul technical debt e sull'incremento della presenza dei code smells. Quindi si andranno a confrontare le classi in cui sono presenti i cloni e le classi in cui i cloni sono assenti e si vedrà come varia il technical debt e i code smell. In questo capitolo introduttivo si andrà a dare una breve definizione di cosa sono i cloni, il technical deb e i code smell.
\section{Clone}
Un clone è una sequenza di istruzioni duplicata in più punti di un codice sorgente quindi, questo significa che è stato effettuato un "copia e incolla" del codice in un altro punto della stessa classe oppure in un'altra classe. La presenza di cloni comporta un grave difetto che consiste nel fatto che se viene individuato un bug in un frammento di codice, questo affligge tutti gli altri frammenti di codice duplicati. La risoluzione di questo problema mediante refactoring potrebbe essere problematica a causa della possibilità di introdurre errori. Esistono però diverse motivazioni che spingono gli sviluppatori ad inserire del codice clonato nei sistemi software, in particolare:
\begin{enumerate}
\item	strategia di sviluppo;
\item	benefici di manutenzione;
\item	superare i limiti linguistici sottostanti;
\end{enumerate}
D'altra parte, però, i cloni del codice possono avere gravi conseguenze sulla qualità, la riusabilità e la manutenibilità di un sistema software. In particolare, si ha:
\begin{enumerate}
\item	aumento della probabilità di propagazione dei bug;
\item	aumento della probabilità di introdurre un nuovo bug;
\item	aumento della probabilità di cattiva progettazione;
\item	aumento della difficoltà nel miglioramento/modifica del sistema;
\item	aumento dei costi di manutenzione;
\item	aumento dei requisiti di risorse.
\end{enumerate}
Si hanno quattro tipologie di cloni suddivise in due aree:
\begin{enumerate}
\item Classificazione basata sulla somiglianza testuale:
\begin{itemize}
\item	Tipo I: frammenti di codice identico tranne che per le variazioni negli spazi bianchi (potrebbero anche essere variazioni nel layout) e nei commenti. Non è possibile usare una tecniche di confronto linea per linea per individuarlo.
\item	Tipo II: frammenti strutturalmente/sintatticamente identici ad eccezione delle variazioni di identificatori, letterali, tipi, layout e commenti. I cloni di questo tipo hanno la stessa struttura sintattica.
\item	Tipo III: frammenti copiati con ulteriori modifiche. Le dichiarazioni possono essere modificate, aggiunte o rimosse in aggiunta alle variazioni di identificatori, letterali, tipi, layout e commenti.
\end{itemize}
\item Classificazione basata sulla somiglianza funzionale: se le funzionalità dei due frammenti di codice sono identiche o simili, cioè hanno condizioni pre e post simili, si parla di cloni semantici e questi vengono identificati come cloni di tipo IV.
\begin{itemize}
\item	Tipo IV: due o più frammenti di codice che eseguono lo stesso calcolo ma implementati attraverso diverse varianti sintattiche. 
\end{itemize}
\end{enumerate}
La complessità analitica e la sofisticazione nel rilevare i cloni aumenta dal Tipo I al Tipo IV. Il rilevamento dei cloni di tipo IV è quindi il più difficile.\\
Sebbene in generale ci siano solo quattro tipi di cloni, vengono usati termini diversi quando ci si riferisce ai cloni stessi:
\begin{itemize}
\item	exact clones: per fare riferimento ai frammenti di codice identici. Tipicamente clone di tipo I;
\item	near-Miss Clones: per fare riferimento a frammenti di codice identici con dichiarazioni aggiunte, cancellate e/o modificate. Fondamentalmente, tutti i cloni di Tipo II sono cloni near-miss;
\item	renamed clones: quando nomi di identificatori, valori letterali, commenti o spazi bianchi modifiche nei frammenti copiati. Essenzialmente un clone di tipo II;
\item	parametrized clone (o p-match): è un clone rinominato con sistematica ridenominazione. Un sottoinsieme di cloni di tipo II.
\end{itemize}
L'analisi dei cloni è stata effettuata mediante l'uso del software Nicad che consente l'estrazione dei cloni di tipo I, II e III. L'output ottenuto da Nicad è stato manipolato attraverso un parser scritto in python per estrapolare solo le informazioni d'interesse.



\section{Technical Debt}
Il technical debt, conosciuto anche come code debt, è la spesa totale che un'organizzazione paga a causa di un'architettura inadeguata o di processi di sviluppo del software inefficienti. Se il debito non viene risolto continua ad accumulare interesse rendendo così più problematico implementare i cambiamenti futuri. Diversi fattori rendono sempre più difficile la gestione o l'aggiornamento del codice sorgente, inclusa la complessità dell'architettura e le pratiche di sviluppo. I fattori che contribuiscono all'incremento del technical debt sono:
\begin{itemize}
\item pressioni aziendali;
\item processi insufficienti;
\item funzioni hard-coded;
\item insiemi di test scadenti;
\item documentazione di codice scadente;
\item mancanza di collaborazione;
\item refactoring differito.
\end{itemize}
Le date di rilascio aggressive fanno sì che le modifiche vengano ignorate o messe in attesa ed inoltre, le funzioni hard-coded creano applicazioni non flessibili che è impossibile aggiornare nei tempi previsti. Codice scarsamente documentato, test insufficienti e comunicazione minima aumentano il technical debt in quanto il team deve dedicare più tempo per trovare o per risolvere problemi dopo il rilascio.\\ 
Lo strumento utilizzato per analizzare il technical debt è SonarCloud cioè la versione online di SonarQube. SonarQube, e di conseguenza SonarCloud, utilizza il metodo SQALE per calcolare il technical debt. Tale metodo è basato esclusivamente su regole e questo significa che se si vuole gestire tutto il technical debt con SQALE, bisogna prima abilitare le regole nel repository Common SonarQube che contrassegna:
\begin{itemize}
\item blocchi duplicati;
\item casi di test falliti;
\item numero di casi di test insufficiente a coprire i branch;
\item densità di commenti insufficiente;
\item numero di test insufficiente a coprire le linee di codice;
\item casi di test saltati.
\end{itemize}
Tali regole si trovano nel repository Common SonarQube perché sono comuni a tutte i linguaggi di programmazione. Una volta abilitate è possibile tenere traccia di tutti i difetti di qualità e monitorare il debito tecnico, che il metodo SQALE misura in giorni.
\subsection{Il metodo Sqale}
Il metodo SQALE è stato sviluppato per rispondere a un'esigenza generica e permanente di valutazione della qualità del codice sorgente visto che standard come ISO 9126 e ISO / IEC 15939 non forniscono un supporto efficace. Si tratta di un metodo generico ed indipendente da linguaggio e strumenti. Esso permette di:
\begin{itemize}
\item definire chiaramente cosa crea il debito tecnico;
\item stimare correttamente il debito;
\item analizzare tale debito dal punto di vista tecnico e di business;
\item offrire strategie di prioritizzazione diverse che consentono di stabilire un piano di ammortamento ottimale. 
\end{itemize}
Il modello di qualità SQALE viene utilizzato per formulare e organizzare i requisiti non funzionali relativi alla qualità del codice. È organizzato in tre livelli gerarchici:
\begin{enumerate}
\item il primo livello è composto da caratteristiche;
\item il secondo di sotto-caratteristiche;
\item il terzo livello è composto da requisiti relativi agli attributi interni del codice sorgente.
\end{enumerate}
Tali requisiti di solito dipendono dal contesto e dalla lingua del software. Qualsiasi violazione di questi requisiti induce il debito tecnico.
\section{Code Smell}
Il termine Code Smell è stato coniato per la prima volta da Kent Beck e nell’Ingegneria del Software indica codice sorgente che, nonostante sia funzionante, presenta delle imperfezioni logiche che ne riducono la qualità. Non si tratta di veri e propri errori ma la loro presenza può rallentare la manutenzione del software e rendere difficile la  scoperta di bug. \\
Quando ci si accinge a progettare un nuovo software è conveniente definirne una struttura facilmente comprensibile, in maniera tale da garantire flessibilità in caso di aggiunte al codice da effettuare in un secondo momento. I Code Smells incrementano il Software decay, cioè il decadimento del Software a causa della poca manutenzione effettuabile su di esso. La metodologia che permette la correzione di questi prende il nome di Refactoring. 
%I code smell sono parti di codice sorgente caratterizzate da difetti di programmazione; non sono definibili come veri e propri errori. La loro presenza non causa problemi direttamente al funzionamento del software, però possono rallentare la manutenzione del software e rendere difficile la  scoperta di bug. 
SonarCloud consente, in base alle regole settate nel quality profile, di individuare diverse tipologie di smells come ad esempio:
\begin{itemize}
\item "==" and "!=" should not be used when "equals" is overridden
\item "@CheckForNull" or "@Nullable" should not be used on primitive types
\item "@Deprecated" code should not be used
\item "@EnableAutoConfiguration" should be fine-tuned
\item "@Import"s should be preferred to "@ComponentScan"s
\item "@Override" should be used on overriding and implementing methods
\end{itemize}
Ad ognuno viene associato un diverso tipo di severità, in particolare si ha:
\begin{enumerate}
\item blocker;
\item major;
\item info;
\item critical;
\item minor.
\end{enumerate}
In base alla severità viene associato un certo Technical Debt.
L'elaborato seguirà la seguente struttura:\\
\autoref{cap3}: illustra i tools utilizzati per effettuare l'analisi.\\
\autoref{cap4}: illustra l'elaborazione dei dati e la relativa analisi.\\
\autoref{cap5}: illustra le conclusioni dell'analisi effettuata.