\section{Technical Debt}
Il technical debt, conosciuto anche come code debt, è la spesa totale che un'organizzazione paga a causa di un'architettura inadeguata o di processi di sviluppo del software inefficienti. Se il debito non viene risolto, continua ad accumulare interesse, rendendo così più problematico implementare i cambiamenti futuri. Diversi fattori rendono sempre più difficile la gestione o l'aggiornamento del codice sorgente, inclusa la complessità dell'architettura e le pratiche di sviluppo. I fattori che contribuiscono all'incremento del technical debt sono:
\begin{itemize}
\item pressioni aziendali;
\item processi insufficienti;
\item funzioni hard-coded;
\item insiemi di test scadenti;
\item documentazione di codice scadente;
\item mancanza di collaborazione;
\item refactoring differito.
\end{itemize}
Le date di rilascio aggressive fanno sì che le modifiche vengano ignorate o messe in attesa ed inoltre, le funzioni hard-coded creano applicazioni non flessibili che è impossibile aggiornare nei tempi previsti. Codice scarsamente documentato, test insufficienti e comunicazione minima aumentano il technical debt in quanto il team deve dedicare più tempo a trovare o risolvere problemi dopo il rilascio.\\ 
Lo strumento utilizzato per analizzare il technical debt è SonarCloud cioè la versione online di SonarQube. SonarQube, e di conseguenza SonarCloud, utilizza il metodo SQALE per calcolare il technical debt. Tale metodo è basato esclusivamente su regole e questo significa che se si vuole gestire tutto il technical debt con SQALE, bisogna prima abilitare le regole nel repository Common SonarQube che contrassegna:
\begin{itemize}
\item blocchi duplicati;
\item casi di test falliti;
\item numero di casi di test insufficiente a coprire i branch;
\item densità di commenti insufficiente;
\item numero di test insufficiente a coprire le linee di codice;
\item casi di test saltati.
\end{itemize}
Tali regole si trovano nel repository Common SonarQube perché sono comuni a tutte le lingue. Una volta abilitate è possibile tenere traccia di tutti i difetti di qualità e monitorare il debito tecnico, che il metodo SQALE misura in giorni.
\subsection{Il metodo Sqale}
Il metodo SQALE è stato sviluppato per rispondere a un'esigenza generica e permanente di valutazione della qualità del codice sorgente visto che standard come ISO 9126 e ISO / IEC 15939 non forniscono un supporto efficace. Si tratta di un metodo generico ed indipendente da linguaggio e strumenti particolarmente dedicato alla gestione del debito tecnico (o del debito di progettazione) degli sviluppi del software. Esso permette di:
\begin{itemize}
\item definire chiaramente cosa crea il debito tecnico;
\item stimare correttamente il debito;
\item analizzare tale debito dal punto di vista tecnico e di business;
\item offrire strategie di prioritizzazione diverse che consentono di stabilire un piano di ammortamento ottimale. 
\end{itemize}
Il modello di qualità SQALE viene utilizzato per formulare e organizzare i requisiti non funzionali relativi alla qualità del codice. È organizzato in tre livelli gerarchici:
\begin{enumerate}
\item il primo livello è composto da caratteristiche;
\item il secondo di sotto-caratteristiche;
\item il terzo livello è composto da requisiti relativi agli attributi interni del codice sorgente.
\end{enumerate}
Tali requisiti di solito dipendono dal contesto e dalla lingua del software. Qualsiasi violazione di questi requisiti induce il debito tecnico.