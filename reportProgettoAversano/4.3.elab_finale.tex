\section{Elaborazione Finale}
Una volta prelevati i dati da Nicad e SonarCloud, all'interno di due file CSV riassuntivi, si è rivelato necessario procedere ad un'opportuna aggregazione delle informazioni. Tale procedimento è stato attuato attraverso un parser\footnote{ParserFinale.py}. L'output generato dal parser è un file CSV che contiene tutti i cloni e tutte le classi di tutte le versioni di un sistema analizzato. L'associazione in un unico file di informazioni diverse, e talvolta ripetute, si è rivelata indispendabile per le varie analisi da effettuare. In particolare, all'interno vengono ripetuti i valori di Technical Debt e Code Smell per ogni clone che insiste su una stessa classe. Tale ripetizione aggiunge alcune importanti informazioni che consentono di effettuare un'analisi più approfondita nel prossimo capitolo, senza influire sulle statistiche estratte.
L'ultimo passo dell'elaborazione sfrutta un programma\footnote{ContatoreClassi.py} che consente di effettuare un conteggio dei file coinvolti nell'analisi dei cloni fornita da Nicad. In particolare procede a verificare la quantità di cloni presenti in un unico file e associa un valore 1 o 0 ad ogni in clone. Questo valore, chiamato NewPath, serve ad indicare se il file che contiene il clone è già apparso, e quindi contiene altri cloni, oppure no.
L'ultima fase del processo di elaborazione dei dati è stata quella dell'aggregazione dei valori ottenuti sui singoli file al fine di effettuare un'analisi statistica. In particolare si è proceduto al calcolo delle varie medie utilizzate per l'analisi complessiva del sistema.
