\section{Elaborazione Dati Nicad}
Nicad è in grado di fornire una grande varietà di dati, alcuni dei quali non sono stati presi in considerazione. Il principale obiettivo dello studio era, infatti, quello di verificare la quantità di cloni all'interno di alcuni progetti open source, e comprendere come questi impattano negativamente sulla qualità del software.
La versione di Nicad utilizzata per le analisi è la 5.0, rilasciata durante l'Ottobre del 2018. 
Il tool è stato invocato a linea di comando attraverso il comando:
\begin{verbatim}
./nicad functions java directory/Sistema type1
\end{verbatim}
Si nota che:
\begin{itemize}
	\item il primo parametro indica il livello di granularità a cui eseguire l'analisi;
	\item il secondo indica il linguaggio in cui è scritto il sistema sotto analisi, che in tutti gli esempi considerati è Java;
	\item segue il path del progetto e il file di configurazione da utilizzare nell'analisi. Quest'ultimo serve a specificare il tipo di clone da individuare.
\end{itemize}

A seguito di successive invocazioni dello comando sullo stesso progetto, si sono ottenuti una serie di file XML. In particolare, si è scelto di effettuare un'analisi per ognuno dei tre tipi di cloni conosciuti, sfruttando tre file di configurazione già presenti in nicad. Questi file specificano il tipo di clone da individuare, andando a modificare in maniera opportuna alcuni dei parametri che sfrutta il tool nella sua analisi. Nicad ad ogni invocazione produce due file XML:
\begin{itemize}
	\item uno di questi contiene le coppie di cloni, accompagnate da tutte le informazioni necessarie alla loro individuazione: un ID, la starline, la endline, il grado di similarità tra i due, ed altre;
	\item l'altro file contiene lo stesso tipo di informazioni, ma riunisce i cloni per classi e non per singole coppie.
\end{itemize}
Ricapitolando: ogni invocazione del tool sul sistema da analizzare, esso produce un file XML contenente tutte le classi di cloni di un particolare tipo, per questo motivo ai fini delle analisi di interesse, si è proceduto ad una triplice invocazione su ogni versione di ogni progetto analizzato, una per ogni tipo di clone presente nella versione.
A partire da questa collezione di file XML si è proceduto a costruire un file CSV che fosse in grado di riassumere tutte le informazioni desiderate. Tale operazione è stata effettuata attraverso un parser ad hoc, ParserXML.py, che preleva le informazioni d'interesse. In particolare sono state selezionate le seguenti informazioni riguardo al clone:
\begin{itemize}
\item l'identificativo della classe
\item l'identificativo del singolo clone
\item il path del file
\item la startline
\item la endline
\item e la similarità
\end{itemize} 
\'E stata aggiunta anche la versione del sistema analizzato e il tipo di clone. 
Il secondo parser utilizzato, ParserCSV.py, serve ad eliminare gli errori di classificazione che commette Nicad nel corso delle sue analisi. Quest'ultimo infatti tende a classificare alcuni cloni come afferenti a più tipi. Ad esempio, può classificare un clone di tipo 1 anche come clone di tipo 2 e 3. Volendo evitare la loro ripetizione all'interno del file CSV, si è proceduto ad utilizzare il ParserCSV.py, il quale identifica i cloni ripetuti e li elimina dal tipo superiore. Ad esempio, se un clone viene identificato come tipo 1 e 2, il parser procede all'eliminazione del clone di tipo 2, conservando quelli di tipo 1.