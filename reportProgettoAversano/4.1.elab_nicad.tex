\section{Elaborazione Dati Nicad}
Le informazioni relative al codice clonato sono state estratte mediante l'uso di Nicad. Il tool è stato invocato a linea di comando attraverso il comando:
\begin{verbatim}
./nicad functions java directory/Sistema type1
\end{verbatim}
Si nota che:
\begin{itemize}
	\item il primo parametro indica il livello di granularità a cui eseguire l'analisi;
	\item il secondo parametro indica il linguaggio in cui è scritto il sistema sotto analisi;
	\item il terzo parametro specifica il path del progetto
	\item il quarto parametro indica il file di configurazione da utilizzare nell'analisi.
\end{itemize}
Il file di configurazione consente di specificare il tipo di clone da individuare. Facendo variare il file di configurazione è stato possibile ottenere, come output, dei file XML relativi ai diversi tipi di cloni. Nicad ad ogni invocazione produce due file XML:
\begin{itemize}
	\item il primo contiene le coppie di cloni accompagnate da una serie di informazioni tra cui: un ID, la startline, la endline, etc;
	\item il secondo file contiene lo stesso tipo di informazioni, ma riunisce i cloni per classi e non per singole coppie.
\end{itemize}
A partire da questa collezione di file XML si è proceduto a costruire un file CSV che fosse in grado di riassumere tutte le informazioni desiderate. Tale operazione è stata effettuata attraverso un parser\footnote{ParserXML.py} che preleva le informazioni d'interesse, in particolare sono state estratte le seguenti informazioni:
\begin{itemize}
\item l'identificativo della classe;
\item l'identificativo del singolo clone;
\item il path del file;
\item la startline;
\item la endline;
\item e la similarità.
\end{itemize} 
È stata aggiunta anche la versione del sistema analizzato e il tipo di clone. 
È stato creato un ulteriore parser\footnote{ParserCSV.py} che consente di rimuovere gli errori di classificazione commessi da Nicad. Quest'ultimo, infatti, tende a classificare alcuni cloni come afferenti a più tipi. Ad esempio, può classificare un clone di tipo 1 anche come clone di tipo 2 e 3. Il parser identifica i cloni ripetuti e li elimina dal tipo superiore. Ad esempio, se un clone viene identificato come tipo 1 e 2 il parser procede all'eliminazione del clone di tipo 2.