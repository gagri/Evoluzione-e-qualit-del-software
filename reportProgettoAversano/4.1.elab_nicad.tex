% !TeX spellcheck = en_GB
\section{Elaborazione Dati Nicad}
Nicad e' in grado di fornire una grande varieta' di dati, alcuni dei quali non sono stati presi in considerazione. Il principale obiettivo dello studio era, infatti, quello di verificare la quantita' di cloni all'interno di alcuni progetti open source, e comprendere, come questi, impattano negativamente sulla qualita' del software.
La versione di Nicad utlizzata per le analisi e' la 5.0, rilasciata durante l'Ottobre del 2018. 
Il tool Nicad e' stato invocato a linea di comando attraverso il comando:

./nicad functions java directory/Sistema type1
\\

Dove il primo parametro indica il livello di granularita' a cui eseguire l'analisi. Il secondo indica il linguaggio in cui e' scritto il sistema sotto analisi, negli esempi sono stati utilizzati solo progetti basati su java. A seguito della directory del sistema vi e' il file di configurazione da utilizzare nell'analisi. In particolare consente di selezionare, ad esempio, il tipo di cloni da individuare.

A seguito di successive invocazioni di questo comando sullo stesso progetto si sono ottenuti una serie di file XML. In particolare si e' scelto di effettuare un'analisi per ognuno dei tre tipi di cloni conosciuti, sfruttando tre file di configurazione gia' presenti in nicad. Questi file specificano il tipo di clone da individuare, andando a modificare in maniera opportuna alcuni dei parametri che sfrutta nicad nella sua analisi. Nicad ad ogni invocazione produce due file XML, uno che contiene le coppie di cloni, accompagnate da tutte le informazioni necessarie alla loro individuazione: un ID, la starline, la endline, il grado di similiarita' tra i due, ed altre. Il secondo file contiene lo stesso tipo di informazioni, ma riunisce i cloni per classi e non per singole coppie.
\\
Ricapitolando ogni invocazione di Nicad sul sistema da analizzare produce un file XML contenente tutte le classi di cloni di un particolare tipo, per questo motivo ai fini delle analisi di interesse, si e' proceduto ad una triplice invocazione di Nicad su ogni versione di ogni progetto analizzato, una per ogni tipo di clone presente nella versione.
A partire da questa collezione di file XML si e' proceduto a costruire un file CSV che fosse in grado di riassumere tutte le informazioni desiderate. Tale operazione e' stata effettuata attraverso un parser ad hoc, ParserXML.py, che preleva le informazioni d'interesse. In particolare sono state selezionate le seguenti informazioni riguardo al clone: l'identificativo della classe, l'identificativo del singolo clone, il path del file, la startline, la endline e la similarita'. A queste il parser, scritto in python, aggiunge la versione del sistema analizzato e il tipo di clone. 
Il secondo parser utilizzato, ParserCSV.py, serve a eliminare gli errori di classificazione che commette Nicad nel corso delle sue analisi. Quest'ultimo infatti tende a classificare alcuni cloni come afferenti a piu' tipi. Ad esempio puo' classificare un clone di tipo 1, anche come clone di tipo 2 e tipo  3. Volendo evitare la loro ripetizione all'interno del file CSV, si e' proceduto ad utilizzare il ParserCSV.py, il quale identifica i cloni ripetuti e li elimina dal tipo superiore. Ad esempio se un clone viene identificato come tipo 1 e tipo 2, il parser procede all'eliminazione del clone di tipo 2, conservando quelli di tipo 1.