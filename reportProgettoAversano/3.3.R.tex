\section{R}
R è un linguaggio ed un ambiente per il calcolo statistico che consente anche il plot dei risultati sotto forma di grafici. Fornisce un'ampia varietà di tecniche statistiche, come:
\begin{itemize}
\item modellazione lineare e non lineare,
\item test statistici classici,
\item analisi di serie temporali,
\item classificazione,
\item clustering,
\item ...
\end{itemize}
Uno dei punti di forza di R è la facilità con cui è possibile produrre grafici di qualità e ben progettati, compresi simboli matematici e formule dove necessario. R è disponibile come Free Software sotto i termini della GNU General Public License della Free Software Foundation sotto forma di codice sorgente. Compila e gira su una vasta gamma di piattaforme UNIX e sistemi simili (inclusi FreeBSD e Linux), Windows e MacOS. Include:
\begin{itemize}
\item un efficace sistema di gestione e stoccaggio dei dati;
\item una suite di operatori per calcoli su array, in particolare matrici;
\item una raccolta ampia, coerente e integrata di strumenti intermedi per l'analisi dei dati;
\item strutture grafiche per l'analisi dei dati e visualizzazione su schermo e
\item un linguaggio di programmazione ben sviluppato, semplice ed efficace che include condizionali, cicli, funzioni ricorsive definite dall'utente e strutture di input e output. 
\end{itemize}
Il termine "ambiente" intende caratterizzarlo come un sistema completamente pianificato e coerente, piuttosto che un accrescimento incrementale di strumenti molto specifici e inflessibili, come spesso accade con altri software di analisi dei dati. R, è progettato intorno a un vero linguaggio informatico e consente agli utenti di aggiungere nuove funzionalità definendo nuove funzioni. Per le attività ad alta intensità di calcolo, i codici in C, C++ e Fortran possono essere collegati e richiamati in fase di esecuzione. Gli utenti esperti possono scrivere codice C per manipolare direttamente oggetti R e può essere esteso (facilmente) tramite pacchetti. Ha il proprio formato di documentazione simile a LaTeX, che viene utilizzato per fornire documentazione completa.
