\chapter{Obbiettivo}
L'obbiettivo dell'elaborato consiste nell'analizzare i sistemi riportati in \autoref{tabella_sistemi} e di osservare come la presenza di cloni costituisca un peggioramento che impatta sul technical debt e sulla presenza di code smells.
\begin{table}[h]
\begin{center}
\begin{tabular}{|c|c|}
\hline
Nome Progetto & Versione \\ \hline
Jabref & 4.3 \\ \cline{2-2}
& 4.2\\ \cline{2-2}
& 4.1\\ \cline{2-2}
& 4.0\\ \hline
DnsJava & 2.1.8 \\ \cline{2-2}
& 2.1.7\\ \cline{2-2}
& 2.1.6\\ \cline{2-2}
& 2.1.5\\ \hline
Fastjson & 1.2.50 \\ \cline{2-2}
& 1.2.40\\ \cline{2-2}
& 1.2.30\\ \cline{2-2}
& 1.2.20\\
\hline
\end{tabular}
\end{center}
\caption{Progetti e relative versioni considerati per l'analisi.}
\label{tabella_sistemi}
\end{table}
\\Verrà effettuato un confronto tra le classi contententi cloni e classi in cui non sono presenti cloni e si confronteranno i risultati ottenuti. Gli strumenti usati a supporto dell'analisi sono i seguenti.
\begin{itemize}
\item Nicad4.
\item SonarCloud.
\item Parser python creati ad hoc.
\end{itemize}