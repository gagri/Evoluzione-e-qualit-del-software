\chapter{Obbiettivo}
L'obbiettivo dell'elaborato consiste nell'analizzare i sistemi riportati in \autoref{tabella_sistemi} e di osservare come la presenza di cloni costituisca un peggioramento che impatta sul technical debt e sulla presenza di code smells.
\newcolumntype{C}[1]{>{\centering\arraybackslash}p{#1}}
\begin{table}[h]
\begin{center}
\begin{tabular}{|C{3cm}|C{3cm}|C{3cm}|C{3cm}|}
\hline
Nome Progetto & Versione & Data & LOC\\ \hline
DnsJava & 2.1.8 & 4 Gennaio 2017 & 25k\\ \cline{2-4}
& 2.1.7 & 15 Febbraio 2015 & 25k\\\cline{2-4}
& 2.1.6 & 14 Ottobre 2013 & 24k\\\cline{2-4}
& 2.1.5 & 10 Aprile 2013 & 24k\\\hline
Jabref & 4.3 & 1 Giugno 2018 & 145k\\ \cline{2-4}
& 4.2 & 26 Aprile 2018 & 145k\\ \cline{2-4}
& 4.1& 23 Dicembre 2017 & 141k\\ \cline{2-4}
& 4.0 & 4 Ottobre 2017 & 138k\\ \hline
Fastjson & 1.2.50 & 20 Agosto 2018 & 156k\\ \cline{2-4}
& 1.2.40 & 4 Novembre 2017 & 144k\\ \cline{2-4}
& 1.2.30 & 26 Marzo 2017 & 125k\\ \cline{2-4}
& 1.2.20 & 22 Ottobre 2016 & 116k\\
\hline
\end{tabular}
\end{center}
\caption{Progetti e relative versioni considerati per l'analisi.}
\label{tabella_sistemi}
\end{table}
\\Verrà effettuato un confronto tra classi contenenti cloni e classi in cui non sono presenti cloni e si confronteranno i risultati ottenuti. Gli strumenti usati a supporto dell'analisi sono i seguenti:
\begin{itemize}
\item Nicad4;
\item SonarCloud;
\item Parser python creati ad hoc.
\end{itemize}